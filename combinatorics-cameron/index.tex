\documentclass{book}

\usepackage{amsmath}
\usepackage{amsfonts}
\usepackage{amssymb}
\usepackage{amscd}
\usepackage{amsthm}
\usepackage{graphics}
\usepackage{graphicx}
\usepackage{tikz}
\usepackage{tikz-cd}
\usepackage{bm}

\newtheorem{theorem}{Theorem}
\newtheorem{definition}{Definition}
\newcommand{\bracenom}{\genfrac{\lbrace}{\rbrace}{0pt}{}}

\graphicspath{ {./} }

\title{Combinatorics Cameron - Solutions Manual}
\author{Mingruifu Lin}
\date{September 2025}

\begin{document}

\maketitle

\tableofcontents

\chapter{On Numbers and Counting}

\subsection*{Exercise 1}
The assumption that there is a largest natural number is wrong.

\subsection*{Exercise 2}
Fuh naw.

\subsection*{Exercise 3}
(a)
Clearly, the base case works: $1 > \frac{1}{e}$. Now suppose
$$n! > \left(\frac{n}{e}\right)^n$$
Then
$$(n + 1) n! > (n + 1)\left(\frac{n}{e}\right)^{n}$$
We derive the right hand side:
$$= (n + 1)\left(\frac{n}{n + 1}\right)^{n}\left(\frac{n + 1}{e}\right)^{n}$$
Using the fact that $(1 + \frac{1}{n})^n < e \quad \Rightarrow \quad \frac{1}{(1 + \frac{1}{n})^n} > \frac{1}{e} \quad \Rightarrow \quad \left(\frac{n}{n + 1}\right)^n > \frac{1}{e}$, we replace the middle term to create the inequality
$$> (n + 1)\frac{1}{e}\left(\frac{n + 1}{e}\right)^{n}$$
$$= \left(\frac{n + 1}{e}\right)^{n + 1}$$
Hence induction step completed.

\noindent
(b)
The first inequality is equivalent to
$$\sqrt[n]{n!} < \frac{n + 1}{2}$$
The left side is the geometric mean from 1 to $n$. The right side is the arithmetic mean from 1 to $n$. The theorem tells us that the geometric mean is smaller than the arithmetic mean.

Once again, we use the same strategy as (a).
$$n! < \left(\frac{n + 1}{2}\right)^n$$
$$= \left(\frac{n}{n}\right)^n \left(\frac{n + 1}{2}\right)^n$$
$$= \left(\frac{n + 1}{n}\right)^n \left(\frac{n}{2}\right)^n$$
$$= \left(1 + \frac{1}{n}\right)^n \left(\frac{n}{2}\right)^n$$
$$< e \left(\frac{n}{2}\right)^n$$
Hence proven.

\subsection*{Exercise 4}
No idea.

\subsection*{Exercise 5}
Ambiguously phrased.

\subsection*{Exercise 6}
Verified.

\subsection*{Exercise 7}
Verified.

\subsection*{Exercise 8}
There are $2^n$ configurations of inputs. Let's list them on a single row. Then, for each of the $2^n$ configurations, we have 2 options, either true or false, hence $2^{2^n}$ boolean functions. This is akin to choosing from the subset of $2^n$ configurations those who bear the value "true", and labelling the rest "false".

\subsection*{Exercise 9}
I guess they can start from the existence of the empty set, which gives 0. Then work their way up using this axiom.

\subsection*{Exercise 10}
We can easily use L'Hopital to deduce that $\lim_{n \rightarrow \infty} \frac{g(n)}{f(n)} \geq \lim_{n \rightarrow \infty} \frac{ab^n}{bn^d} = \infty$. This means that it is unbounded, hence we can find $C = 1$ such that $\frac{g(n)}{f(n)} > C = 1$, hence $g(n) > f(n)$. Too lazy to find the equality point.

\subsection*{Exercise 11}
For the first, since $b$ subsets each with $k$ elements, then the combined total count of elements of $bk$. For the second, there are $v$ elements, each found in exactly $r$ subsets, which adds up to $vr$, which is also the combined total count of elements in subsets. Too lazy for examples.

\subsection*{Exercise 12}
(i)
Suppose $c = ab$. If $a$ is even, then
$$c = \frac{a}{2} (2b)$$
If $a$ is odd, then
$$c = ab - b + b$$
$$= (a - 1)b + b$$
$$= \frac{a - 1}{2}(2b) + b$$
At the end of the algorithm after $n$ steps, we have $c = 1b_n + b_{i_1} + b_{i_2} + \cdots + b_{i_m}$, where, as per the instructions, we have $a_n = 1$. The $b_i$ come from the shedding of the odd $a$'s. Hence proven.

\noindent
(ii)
Idk, too lazy.

\noindent
(iii)
Fuh naw.

\subsection*{Exercise 13}
Yeah, Buddha is dumb af.

\subsection*{Exercise 14}
Number of books:
$$B = 25^{410 \cdot 40 \cdot 80}$$
Number of rooms:
$$R = \frac{B}{20 \cdot 35}$$

\subsection*{Exercise 15}
Hell nah.

\end{document}
