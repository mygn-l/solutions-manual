\documentclass{book}

\usepackage{amsmath}
\usepackage{amsfonts}
\usepackage{amssymb}
\usepackage{amscd}
\usepackage{amsthm}
\usepackage{graphics}
\usepackage{graphicx}
\usepackage{tikz}
\usepackage{tikz-cd}
\usepackage{bm}

\newtheorem{theorem}{Theorem}
\newtheorem{definition}{Definition}
\newcommand{\bracenom}{\genfrac{\lbrace}{\rbrace}{0pt}{}}

\graphicspath{ {./} }

\title{Graph Theory Bondy - Solutions Manual}
\author{Mingruifu Lin}
\date{December 2025}

\begin{document}

\maketitle

\tableofcontents

\chapter{Graphs}

\section{Graphs and Their Representation}

\subsection*{Exercise 1.1.1}
$m$ is the number of edges. $\binom{n}{2}$ is the number of pairs of vertices. In a simple graph, every pair of vertices share at most one edge. Hence, $m \leq \binom{n}{2}$.

\subsection*{Exercise 1.1.2}
(a)
Each of the $r$ vertices in $X$ can link to at most $s$ vertices in $Y$. It cannot link to any other vertex. The maximum number of edges is thus $rs$. Hence $m \leq rs$.

\noindent
(b)
$r = \frac{n}{2} + k$ and $s = \frac{n}{2} - k$ for some $k$. For verification, notice that $r + s = n$. Then,
$$rs = \left(\frac{n}{2} + k\right)\left(\frac{n}{2} - k\right)$$
$$= \frac{n^2}{4} - k^2 \leq \frac{n^2}{4}$$
Of course, this only works if $n$ is even, but the odd case follows a similar procedure.

\noindent
(c)
The equality is strict when $k = 0$, i.e. both sides have the same number of vertices.

\subsection*{Exercise 1.1.3}
(a)
Simply alternate sides.

\noindent
(b)
For every edge that goes to the other side, the next edge must come back. This creates pairs of back-and-forth edges. Since the number of edges is even, then the number of vertices is even as well.

\subsection*{Exercise 1.1.4}
It's pretty obvious. $d(v_i) \geq \delta(G)$ for all $i$. Hence
$$\sum_{i = 1}^n d(v_i) \geq n\delta(G)$$
$$\frac{\sum_{i = 1}^n d(v_i)}{n} = d(G) \geq \delta(G)$$
A similar procedure is used for $\Delta(G)$.

\subsection*{Exercise 1.1.5}
When $k = 0$, every vertex is isolated. For $k = 1$, the vertices are grouped in pairs, hence the graphs must have an even number of vertices. For $k = 2$, the graphs consist of disjoint cycles of arbitrary length.

\subsection*{Exercise 1.1.6}
(a)

\end{document}
