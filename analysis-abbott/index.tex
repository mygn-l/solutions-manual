\documentclass{book}

\usepackage{amsmath}
\usepackage{amsfonts}
\usepackage{amssymb}
\usepackage{amscd}
\usepackage{amsthm}
\usepackage{graphics}
\usepackage{graphicx}
\usepackage{tikz}
\usepackage{tikz-cd}
\usepackage{bm}

\newtheorem{theorem}{Theorem}
\newtheorem{definition}{Definition}
\newcommand{\bracenom}{\genfrac{\lbrace}{\rbrace}{0pt}{}}

\graphicspath{ {./} }

\title{Analysis Abbott - Solutions Manual}
\author{Mingruifu Lin}
\date{September 2023}

\begin{document}

\maketitle

\tableofcontents

\chapter{The Real Numbers}

\section{}
\section{Some Preliminaries}

\subsection*{Exercise 1.2.1}
(a)
We have the same equation:
$$a^2 = 3b^2$$
which requires $b$ to contain a factor of $3$, making both divisible by $3$, violating the premise that the fraction is irreducible. With $6$, it's the same thing, we must find the $2$ and $3$ inside $b$.

\noindent
(b)
It breaks down because we have
$$a^2 = 2 \cdot 2 \cdot b^2$$
where we can find the factor of $2$ already, without requiring $b$ to possess it.

\subsection*{Exercise 1.2.2}
$$2^r = 2^{\frac{a}{b}} = 3$$
$$2^a = 3^b$$
Clearly, impossible, no matter what $a, b$ you choose.

\subsection*{Exercise 1.2.3}
(a)
False. Consider $A_n = (-\frac{1}{n}, \frac{1}{n})$. The intersection is $\{0\}$.

\noindent
(b)
True. Notice that $A_m \subseteq A_n$ whenever $m \geq n$, hence $A_m \subseteq A_1$. Also, if $x \in \bigcap A_n$ then, for some $n$, we have $x \in A_n \subseteq A_1$, hence $x \in A_1$. Hence, $\bigcap A_n \subseteq A_1$, so it must be finite since $A_1$ is finite.

For the next part, since each $A_n$ is finite, we can take the maximum of each set, which is guaranteed to exist by being non-empty. The maxima decrease as a sequence, but can only decrease finitely many times, after which the elements must remain identical. Notice that this identical value is contained in each $A_n$, hence contained in their intersection, making the intersection non-empty.

\noindent
(c)
False. Take $A = \varnothing$, and take $B = C = \{ x \}$. The left side is $\varnothing$, while the right side is $\{ x \}$.

\noindent
(d)
True. This can be proved using formal logic.

\noindent
(e)
True. This can be proved using formal logic.

\subsection*{Exercise 1.2.4}
Let $A_1$ be the even numbers with an additional element $1$. Let $A_2$ be the multiples of $3$ that are not multiples of $2$. Let $A_3$ be the multiples of $5$ that are not multiples of $2$ nor $3$. And so on.

\subsection*{Exercise 1.2.5}
This can be proved using formal logic.

\subsection*{Exercise 1.2.6}
(a)
Verified.

\noindent
(b)
$$(a + b)^2 = a^2 + 2ab + b^2 \leq |a|^2 + 2|a||b| + |b|^2 = (|a| + |b|)^2$$
Taking the square root on both ends yields
$$|a + b| \leq ||a| + |b|| = |a| + |b|$$
where we accept that
$$\sqrt{x^2} = |x|$$

\noindent
(c)
$$|a - b|$$
$$= |a - c + c - d + d - b|$$
$$\leq |a - c| + |c - d + d - b|$$
$$\leq |a - c| + |c - d| + |d - b|$$

\noindent
(d)
$$|a| = |a - b + b| \leq |a - b| + |b|$$
$$\Rightarrow |a| - |b| \leq |a - b|$$
By symmetry, $-(|a| - |b|) = |b| - |a| \leq |b - a| = |a - b|$. Hence,
$$||a| - |b|| \leq |a - b|$$

\subsection*{Exercise 1.2.7}
(a)
$f(A) = [0, 4]$ and $f(B) = [1, 16]$.
$$f(A \cap B) = f([1, 2]) = [1, 4] = f(A) \cap f(B)$$
$$f(A \cup B) = f([0, 4]) = [0, 16] = f(A) \cup f(B)$$

\noindent
(b)
Let $A = [-2, -1]$ and $B = [1, 2]$. Then $f(A \cap B) = f(\varnothing) = \varnothing$. But $f(A) = [1, 4]$ and $f(B) = [1, 4]$, so $f(A) \cap f(B) = [1, 4] \neq \varnothing$.

\noindent
(c)
$$\{ g(x) \mid x \in A \cap B \} \subseteq \{ g(x) \mid x \in A \}$$
$$\{ g(x) \mid x \in A \cap B \} \subseteq \{ g(x) \mid x \in B \}$$
$$\Rightarrow \{ g(x) \mid x \in A \cap B \} \subseteq \{ g(x) \mid x \in A \} \cap \{ g(x) \mid x \in B \}$$

\noindent
(d)
Conjecture:
$$g(A \cup B) = g(A) \cup g(B)$$
Proof:
$$\{ g(x) \mid x \in A \} \subseteq \{ g(x) \mid x \in A \cup B \}$$
$$\{ g(x) \mid x \in B \} \subseteq \{ g(x) \mid x \in A \cup B \}$$
$$\Rightarrow \{ g(x) \mid x \in A \} \cup \{ g(x) \mid x \in B \} \subseteq \{ g(x) \mid x \in A \cup B \}$$
Reverse direction:
$$g(x) \in g(A \cup B)$$
$$\Rightarrow x \in A \cup B$$
If $x \in A \Rightarrow g(x) \in g(A)$. Likewise, if $x \in B \Rightarrow g(x) \in g(B)$. Hence, $g(x) \in g(A \cup B) \Rightarrow g(x) \in g(A)$ or $g(x) \in g(B)$, which is the definition of
$$g(A \cup B) \subseteq g(A) \cup g(B)$$

\subsection*{Exercise 1.2.8}
(a)
$f(n) = 2n$

\noindent
(b)
$f(n) = \lfloor \frac{n + 1}{2} \rfloor$

\noindent
(c)
$f(n) = (-1)^n \lfloor \frac{n}{2} \rfloor$.

\subsection*{Exercise 1.2.9}
(a)
$f^{-1}(A) = [0, 2]$ and $f^{-1}(B) = [0, 1]$. We have $f^{-1}(A \cap B) = f^{-1}([0, 1]) = [0, 1]$ and $f^{-1}(A) \cap f^{-1}(B) = [0,1]$, so yes, the intersection is correct. Then we have $f^{-1}(A \cup B) = f^{-1}([-1, 4]) = [0, 2]$ and $f^{-1}(A) \cup f^{-1}(B) = [0,2]$, so yes, the union is correct.

\noindent
(b)
For intersection:
$$x \in g^{-1}(A \cap B)$$
$$\Leftrightarrow g(x) \in A \cap B$$
$$\Leftrightarrow [g(x) \in A] \land [g(x) \in B]$$
$$\Leftrightarrow [x \in g^{-1}(A)] \land [x \in g^{-1}(B)]$$
$$\Leftrightarrow x \in g^{-1}(A) \cap g^{-1}(B)$$
For union:
$$x \in g^{-1}(A \cup B)$$
$$\Rightarrow g(x) \in A \cup B$$
If $g(x) \in A$, then $x \in g^{-1}(A)$, hence $x$ is in whatever unions with that thing, so $x \in g^{-1}(A) \cup g^{-1}(B)$. The same procedure goes for $g(x) \in B$. For the reverse direction, suppose that
$$x \in g^{-1}(A) \cup g^{-1}(B)$$
If $x \in g^{-1}(A)$, then $g(x) \in A$, so $g(x)$ is in whatever unions with that thing, so $g(x) \in A \cup B$, hence $x \in g^{-1}(A \cup B)$. The same procedure goes for $x \in g^{-1}(B)$.

\subsection*{Exercise 1.2.10}
(a)
False. Consider $a = b $. It is true that $a < b + \epsilon$ for every $\epsilon > 0$. However, it is false that $a < b$.

\noindent
(b)
False, same as part (a).

\noindent
(c)
True. If $a = b$, then $a < b + \epsilon$ for all $\epsilon > 0$. If $a < b$, then $a < b + \epsilon$. For the reverse direction, if $a < b + \epsilon$ for all $\epsilon > 0$, consider $a = b$, which works. Since $a = b$ works, then anything smaller than $b$ also works, i.e. $a < b$ works. Hence $a \leq b$.

\subsection*{Exercise 1.2.11}
(a)
There exists real numbers $a < b$ such that $a + \frac{1}{n} \geq b$ for all $n \in \mathbb{N}$.

\noindent
(b)
For all $x > 0$, there exists $n \in \mathbb{N}$ such that $x \geq \frac{1}{n}$.

\noindent
(c)
There exists two distinct real numbers such that there is no rational number between them.

\subsection*{Exercise 1.2.12}
(a)
$$y_1 = 6 > -6$$
so the base case is true. Now, suppose $y_n > -6$. Then
$$y_{n + 1} = \frac{2y_n - 6}{3} = \frac{2}{3}y_n - 2 > \frac{2}{3}(-6) - 2 = -6$$
hence the induction step is complete.

\noindent
(b)
$y_2 = 2$ and $y_1 = 6$ so the base case is true. Now suppose $y_n > y_{n + 1}$, then
$$2y_n > 2y_{n + 1}$$
$$\Rightarrow 2y_n - 6 > 2y_{n + 1} - 6$$
$$\Rightarrow \frac{2y_n - 6}{3} > \frac{2y_{n + 1} - 6}{3}$$
where no sign-flip occurs because the factors are positive.
$$\Rightarrow y_{n + 1} > y_{n + 2}$$

\subsection*{Exercise 1.2.13}
(a)
$(A_1)^c = A_1^c$ so the base case is true. Now, suppose $\left( \bigcup_{i = 1}^n A_i \right)^c = \bigcap_{i = 1}^n A_i^c$, then
$$\left( \bigcup_{i = 1}^{n + 1} A_i \right)^c$$
$$= \left( \left(\bigcup_{i = 1}^n A_i \right) \cup A_{n + 1} \right)^c$$
$$= \left(\bigcup_{i = 1}^n A_i \right)^c \cap A_{n + 1}^C$$
$$= \left( \bigcap_{i = 1}^n A_i^c \right) \cap A_{n + 1}^C$$
$$= \left( \bigcap_{i = 1}^{n + 1} A_i^c \right)$$
hence the induction step is complete.

\noindent
(b)
$B_n = (0, \frac{1}{n})$

\noindent
(c)
$$x \in \left( \bigcup_{i = 1}^n A_i \right)^c$$
$$\Rightarrow x \not \in \bigcup_{i = 1}^n A_i$$
Suppose for contradiction that $x \in A_i$ for some $i$, then it implies that $x \in \bigcup_{i = 1}^n A_i$, which contradicts our assumption. Hence, $x \not \in A_i$ for all $i$, or equivalently, $x \in A_i^c$ for all $i$. Hence,
$$\Rightarrow x \in \bigcap_{i = 1}^n A_i^c$$
For the reverse direction,
$$x \in \bigcap_{i = 1}^n A_i^c$$
$$\Rightarrow (\forall i) x \not \in A_i$$
Suppose for contradiction that
$$x \not \in \left( \bigcup_{i = 1}^n A_i \right)^c$$
$$\Rightarrow x \in \bigcup_{i = 1}^n A_i$$
$$\Rightarrow (\exists i) x \in A_i$$
which contradicts our assumption. Hence,
$$x \in \left( \bigcup_{i = 1}^n A_i \right)^c$$
The equality is thus proved.

\section{The Axiom of Completeness}

\subsection*{Exercise 1.3.1}
(a)
Given a set $A$ that is bounded below, the infimum of $A$ is a real number $x$ such that
\begin{enumerate}
    \item $x$ is a lower bound of $A$
    \item $x \geq y$ for every lower bound $y$ of $A$
\end{enumerate}

\noindent
(b)
Too lazy

\subsection*{Exercise 1.3.2}
(a)
$B = \{1\}$

\noindent
(b)
Impossible.

\noindent
(c)
$B = \{ \frac{1}{n} \mid n \in \mathbb{N} \}$

\subsection*{Exercise 1.3.3}
(a)
$\sup B$ satisfies the second criterion, since by definition of $\sup B$ first property, we know $\sup B \geq b$ for all $b \in B$, which are precisely all the lower bounds $b$ of $A$. Also, by definition of $\sup B$ second property, we know $\sup B \leq a$ for all upper bounds $a$, which are precisely a superset of $A$. Hence, proven.

\noindent
(b)
Supremum implies infimum exists, since we can construct and define it from the supremum.

\subsection*{Exercise 1.3.4}
(a)
$$\sup \left( \bigcup_{k = 1}^n A_k \right) = \max(\sup A_1, \sup A_2, \ldots, \sup A_n)$$

\noindent
(b)
No, we would use
$$\sup \left( \bigcup_{k = 1}^\infty A_k \right) = \sup(\{ \sup A_k \mid k \in \mathbb{N} \})$$

\subsection*{Exercise 1.3.5}
(a)
For all $a \in A$,
$$\sup A \geq a$$
$$\Rightarrow c \sup A \geq ca$$
hence $c \sup A$ is an upperbound of $cA$. Also, every upper bounds of $cA$ is of the form $cb$ for some upper bound $b$ of $A$.
$$\sup A \leq b$$
$$\Rightarrow c \sup A \leq cb$$
hence $c \sup A$ is smaller or equal to every upper bound of $cA$. Hence, the two criteria are satisfied, making $\sup(cA) = c \sup A$.

\noindent
(b)
$$\sup(cA) = c\inf(A)$$

\subsection*{Exercise 1.3.6}
(a)
$s \geq a$ and $t \geq b$ for all $a, b \in A, B$. Hence $s + t \geq a + b$, hence $s + t$ is an upper bound of $A + B$.

(b) (c) (d) Too lazy

\subsection*{Exercise 1.3.7}
It already satisfies the property of being an upper bound. Now, for every upper bound $b$ of $A$, we have $b \geq x$ for all $x \in A$, which includes $x = a$ since $a \in A$, which gives the inequality: $b \geq a$. This means that every upper bound of $A$ is greater or equal to $a$, completing the second property of supremum.

\subsection*{Exercise 1.3.8}
(a)
Infimum: 0. Supremum: 1.

\noindent
(b)
Infimum: -1. Supremum: 1.

\noindent
(c)
Infimum: $\frac{1}{4}$. Supremum: $\frac{1}{3}$.

\noindent
(d)
Infimum: 0. Supremum: 1.

\subsection*{Exercise 1.3.9}
(a)
Either $\sup B \in B$ or $\sup B \not \in B$. In the first case, we simply take $b = \sup B$. This works because $\sup B > \sup A > a$ for all $a \in A$, hence $\sup B$ is the upper bound for which we were looking. In the second case, notice that there exist $b \in B$ arbitrarily close to $\sup B$. Suppose for contradiction that this is not the case, i.e. there is a distance $\epsilon > 0$ empty zone around the supposed $\sup B$. Then we can simply take a number between $\sup B - \epsilon$ and $\sup B$. This number is an upper bound of $B$ smaller than $\sup B$, contradicting the second property of supremum. Hence, there always exists $b \in B$ such that $|b - \sup B| < \frac{1}{n}$ for all $n \in \mathbb{N}$. To complete the proof, since $\sup A < \sup B$, we set $\frac{1}{n} \approx |\sup B - \sup A|$, and find $b \in B$ within this distance to be our upper bound, which is guaranteed to exist.

\noindent
(b)
Consider $A = \{ 1 \}$ and $B = \{ 1 - \frac{1}{n} \mid n \in \mathbb{N} \}$. Here, $\sup A = \sup B = 1$, but none of the elements of $B$ is greater than $1 \in A$.

\subsection*{Exercise 1.3.10}
(a)
Let $c = \sup A$. It exists because $A$ is non-empty and each element of $A$ is bounded above by an element of $B$ non-empty as well. By first property of supremum, $c \geq a$ for all $a \in A$. By the second property of supremum, $c \leq b$ for all upper bounds $b$ of $A$. By definition, $B$ happens to be precisely the set of upper bounds of $A$, hence $c \leq b$ for all $b \in B$. Thus, $\sup A$ satisfies the requirement for $c$ in the cut property.

\noindent
(b)
Let $E$ be a non-empty set bounded above. Let $B$ be the set of upper bounds of $E$. Also, let $A$ be the set of lower bounds of $B$. By the cut property (it is easy to verify that these sets satisfy the criteria), there exists $c$ such that $c \geq a$ and $c \leq b$ for all $a, b \in A, B$. Now, already, $c$ satisfies the second property of supremum of $E$, since it's smaller or equal to all upper bounds of $E$. Next, since for every $e \in E$, we have $e \leq b$ for all $b \in B$ (by definition of $B$), i.e. every $e$ is precisely a lower bound of $B$, hence we can claim $E \subseteq A$. Thus, $c \geq e$ for every $e \in E$, completing the proof.

\noindent
(c)
Again, we can just take $A = \{ a \mid a^2 < 2 \}$ and $B = \{ b \mid b^2 > 2 \}$. Clearly, $c = \sqrt{2}$ does not exist.

\subsection*{Exercise 1.3.11}
(a)
True. Suppose for contradiction that $\sup B < \sup A$, then we can find $a \in A$ arbitarily close to $\sup A$ such that $a > \sup B \geq b$, for all $b \in B$. But if $a > b$ for all $b \in B$, this implies $a$ is different from every $b \in B$, hence contradicting the assumption that $A \subseteq B$.

\noindent
(b)
True. Between every two distinct real numbers, we can find another distinct real number. Call it $c$. By definition, $c > \sup A \geq a$ and $c < \inf B \leq b$ for all $a, b \in A, B$.

\noindent
(c)
False. Take $A = \{ \frac{1}{n} \mid n \in \mathbb{N} \}$ and $B = \{ -\frac{1}{n} \mid n \in \mathbb{N} \}$. Clearly, $c = 0$ satisfies $a < c < b$ for every $a, b \in A, B$, but $\sup B = \inf A = 0$.

\section{Consequences of Completeness}

\subsection*{Exercise 1.4.1}
(a)
$$ab = \left(\frac{p}{q}\right) \left(\frac{m}{n}\right) = \frac{pm}{qn}$$
Clearly, $pm$ and $qn$ are integers, hence the fraction is a rational number.
$$a + b = \frac{p}{q} + \frac{m}{n} = \frac{pn + mq}{qn}$$
Clearly, $pn + mq$ and $qn$ are integers, hence the fraction is a rational number.

\noindent
(b)
Suppose for contradiction that $a + t = b \in \mathbb{Q}$. We can rearrange the equation as $b - a = t$. In part (a), we showed that the sum of two rationals is rational, hence this would make $t$ rational as well, contradicting our assumption. Hence $a + t$ is irrational.

Suppose for contradiction that $at = b \in \mathbb{Q}$. We can rearrange the equation as $\frac{b}{a} = t$. In part (a), we showed that the product of two rationals is rational, hence this would make $t$ rational as well, contradicting our assumption. Hence, $at$ is irrational.

\noindent
(c)
No, irrationals are not closed under addition and multiplication. For example, if $s = 1 - \sqrt{2}$ and $t = 1 + \sqrt{2}$, by the previous part (b), they are irrational, but their sum is $2$, which is rational. The same thing happens with multiplication, such as $s = t = \sqrt{2}$.

\subsection*{Exercise 1.4.2}
For the first property of supremum, suppose $s$ is not an upper bound of $A$. Then there exists $a \in A$ such that $a > s$. By the Archimedean property, we can find $\frac{1}{n} < a - s$ (i.e. between $a$ and $s$ we can find an ever smaller distance), thus $s + \frac{1}{n} < a$, contradicting the assumption that $s + \frac{1}{n}$ is an upper bound for all $a \in A$. Hence, $s$ must be an upper bound of $A$.

A similar procedure is used for the second property of supremum.

\subsection*{Exercise 1.4.3}
Clearly, every negative number is not in $(0, \frac{1}{n})$ for every $n$. Zero is also excluded. Now, suppose there exists $x > 0$ such that $x \in \bigcap_{n = 1}^\infty (0, \frac{1}{n})$. However, by the Archimedean property, we can find $\frac{1}{n} < x$, hence there exists $(0, \frac{1}{n})$ which does not contain $x$, contradicting our assumption. Hence, no element exists in this intersection.

\subsection*{Exercise 1.4.4}
Too lazy.

\subsection*{Exercise 1.4.5}
Too lazy.

\subsection*{Exercise 1.4.6}
(a)
Not dense.

\noindent
(b)
Dense.

\noindent
(c)
No dense.

\subsection*{Exercise 1.4.7}
Too lazy

\subsection*{Exercise 1.4.8}
(a)
Let $A = \{ 1 - \frac{1}{n} \mid n = 2m, \quad m \in \mathbb{N} \}$ and $B = \{ 1 - \frac{1}{n} \mid n = 2m - 1, \quad m \in \mathbb{N} \}$.

\noindent
(b)
Let $J_n = (-\frac{1}{n}, \frac{1}{n})$.

\noindent
(c)
Let $L_n = [n, \infty)$.

\noindent
(d)
Impossible. If the intersection is non-empty, then there are some elements that are included in every $I_n$ no matter how far you go.

\section{Cardinality}

\subsection*{Exercise 1.5.1}
Too lazy.

\subsection*{Exercise 1.5.2}
They are either not nested or not bounded.

\subsection*{Exercise 1.5.3}
Too lazy.

\subsection*{Exercise 1.5.4}
(a)
Let $f(x) = \tan\left(\frac{\pi}{b - \mu}(x - \mu)\right)$ where $\mu = \frac{a + b}{2}$.

\noindent
(b)
Let $f(x) = \ln(x - a)$.

\noindent
(c)
Let $f\left(\frac{1}{n}\right) = \frac{1}{n + 1}$ for $n \in \mathbb{N}$. Let $f(0) = \frac{1}{2}$. Let $f(x) = x$ otherwise.

\subsection*{Exercise 1.5.5}
(a)
Let $f(x) = x$.

\noindent
(b)
If $f$ is a bijection, then $f^{-1}$ exists and also a bijection.

\noindent
(c)
$f: A \rightarrow B$ bijection and $g: B \rightarrow C$ bijection implies that $g \circ f: A \rightarrow C$ is a bijection as well.

\subsection*{Exercise 1.5.6}
(a)
Let each open interval be $\left(\frac{1}{n + 1}, \frac{1}{n}\right)$ for $n \in \mathbb{N}$.

\noindent
(b)
Impossible. Each open set contains a rational number, hence the number of disjoint open sets is limited by the countability of rationals.

\subsection*{Exercise 1.5.7}
(a)
Let $f(s) = (s, 0)$.

\noindent
(b)
For each $s \in (0, 1)$, choose a unique decimal expansion $0.s_1t_1s_2t_2\ldots$, where $s, t$ are digits (0 to 9). Some have multiple representations, but just choose something that you like. We can construct a point $(0.s_1s_2\ldots, 0.t_1t_2\ldots)$. Clearly, it's bijective.

\subsection*{Exercise 1.5.8}
For every $\frac{1}{n}$, the number of $b \in B$ such that $b > \frac{1}{n}$ must be finite (because there would be divergent series if that wasn't the case). Separate $B$ into these finite chunks of $B \cap [\frac{1}{n + 1}, \frac{1}{n})$, with the first chunk being $B \cap [1, \infty)$. Notice that every $b \in B$ must fall within such a chunk (because we assumed $b > 0$). Notice also that the number of chunks are countable. Since a countable union of finite sets is countable, then $B$ is countable.

\subsection*{Exercise 1.5.9}
(a)
$\sqrt{2}$ solves the equation $x^2 - 2 = 0$.

\noindent
$\sqrt[3]{2}$ solves the equation $x^3 - 2 = 0$.

\noindent
Too lazy to solve the third one.

\noindent
(b) and (c)
Each algebraic number can be associated with a polynomial. Each polynomial is a finite list of integers. Hence, the algebraic numbers are countable.

\subsection*{Exercise 1.5.10}
(a)
Similar to exercise 1.5.8, assume $C \cap [\frac{1}{n}, 1]$ is countable for every $n \in \mathbb{N}$. Separating into chunks, we see again that $C$ is countable, with the starting $0$ negligible. By contradiction, there must exists a $\frac{1}{n}$ that makes $C \cap [\frac{1}{n}, 1]$ uncountable. Simply set your $a$ to be between $0$ and $\frac{1}{n}$.

\noindent
(b)
No, $\alpha = \sup A$ cannot be in $A$. Suppose for contradiction that $\alpha \in A$, then $C \cap [\alpha, 1]$ would be uncountable. This means that $C \cap [\alpha + \frac{1}{n}, 1]$ would be countable for every $n \in \mathbb{N}$. As we saw previously, the countable union of disjoint countable chunks is countable, therefore $C \cap (\alpha, 1]$ is countable. Appending the missing $\{ \alpha \}$ gives us $\{ \alpha \} \cup (C \cap (\alpha, 1]) = (C \cup \{ \alpha \}) \cap [\alpha, 1]$ which remains countable. This is simply a set with a negligible additional element compared to $C \cap [\alpha, 1]$, hence the latter is also uncountable. Hence proven.

\noindent
(c)
No. Simply choose $C = \{ \frac{1}{n} \mid n \in \mathbb{N} \}$. No matter how close $a$ gets to $0$, there will always be finite terms of $C$ after $a$.

\subsection*{Exercise 1.5.11}
(a)
Well, if we have the partition already, simply define
$$h(x) =
\begin{cases}
    g^{-1}(x) & x \in A \\
    f(x) & x \in A'
\end{cases}$$
which is clearly a bijection.

\noindent
(b)
We use proof by induction.
$$A_1 = X \setminus g(Y)$$
$$f(A_1) \subseteq Y \Rightarrow g(f(A_1)) \subseteq g(Y)$$
Notice that $g(f(A_1)) = A_2$. Here, we see that $A_1$ discards $g(Y)$, while $A_2$ is a subset of $g(Y)$, hence $A_1, A_2$ must be disjoint. So the base case is true.

Now, suppose $A_1, \ldots, A_n$ are pairwise disjoint. Then $f(A_1), \ldots, f(A_n)$ are pairwise disjoint by injectivity of $f$. Then $g(f(A_1)), \ldots, g(f(A_n))$ are pairwise disjoint by injectivity of $g$. But these are precisely $A_2, \ldots, A_{n + 1}$, so these are pairwise disjoint. Also, $A_{n + 1}$ which is contained in $g(Y)$ must be pairwise disjoint with $A_1$ which discards $g(Y)$. Hence, $A_1, \ldots, A_{n + 1}$ are pairwise disjoint, so induction step complete. You can interleave the procedure for showing $f(A_n)$ among these sentences.

\noindent
(c)
This is obvious. Each $A_n$ is mapped to $f(A_n)$, so the disjoint union of $A_n$ is mapped to the disjoint union of $f(A_n)$.

\noindent
(d)
Since $B'$ and $B$ are disjoint, then $g(B')$ and $g(B)$ are disjoint by injectivity of $g$, and their union covers the entire $g(Y)$ portion. Notice that $g(B)$ and $A \setminus g(Y)$ cover $A$ entirely. $A'$ only consists of the remaining $g(Y) \setminus g(B)$, which is thus covered by $g(B')$ by surjection of $g: Y \rightarrow g(Y)$.

\section{Cantor's Theorem}

\subsection*{Exercise 1.6.1}
This is obvious. They have the same cardinality.

\subsection*{Exercise 1.6.2}
(a)
The first digit differs.

\noindent
(b)
The n-th digit differs.

\noindent
(c)
We created a real number that is not bijected with a natural. In fact, for every such supposed bijection, we can find a real number not bijected to a natural, hence there exists not bijection of $(0, 1)$ with the naturals, making it uncountable.

\subsection*{Exercise 1.6.3}
(a)
The constructed number need not be rational.

\noindent
(b)
Too lazy. Idk, but prob not an issue. Gut instincts.

\subsection*{Exercise 1.6.4}
These are binary numbers, which are uncountable.

\subsection*{Exercise 1.6.5}
Too lazy. Same for the next exercises.

\subsection*{Exercise 1.6.9}
The power set of naturals is akin to binary numbers, which are uncountable.

\subsection*{Exercise 1.6.10}
(a)
Countable. This is $\mathbb{N} \times \mathbb{N}$.

\noindent
(b)
Uncountable. These are the binary numbers.

\noindent
(c)
Yes. Partition the naturals into groups of 2. Given a binary number, look at the digit at each position, if the digit is 0, choose the left one in the group, if the digit is 1, choose the right one in the group.

\chapter{Sequences and Series}

\section{}
\section{The Limit of a Sequence}

\subsection*{Exercise 2.2.1}
Any bounded sequence "verconges". For example, $a_n = (-1)^n$, which works for $\epsilon = 1.5$. In fact, the sequence converges to every $x \in \mathbb{R}$.

\subsection*{Exercise 2.2.2}
Hell nah.

\subsection*{Exercise 2.2.3}
(a)
There exists a college in the United States such that all students are less than seven feet tall.

\noindent
(b)
There exists a college in the United States such that all professors give C, D, E, or F, for at least one student.

\noindent
(c)
All colleges in the United States have a student who is less than six feet tall.

\subsection*{Exercise 2.2.4}
(a)
Consider the sequence where you insert zeros after skipping 1 term, then skipping 2 terms, then skipping 3 terms, and so on. The rest is all filled with one's, so you still have infinite one's.

\noindent
(b)
Impossible. I'll transcribe what it means for there to be infinite 1's: For every $n \in \mathbb{N}$, we can find $a_m = 1$ for $m > n$. If it converges to $L \neq 1$, then take $0 < \epsilon < |L - 1|$ and you have a contradiction.

\noindent
(c)
Same as (a).

\subsection*{Exercise 2.2.5}
Hell nah. They are easy to find.

\subsection*{Exercise 2.2.6}
For every $\epsilon > 0$, we can find sufficiently large $m$ such that
$$|a_m - a| < \epsilon$$
$$|a_m - b| < \epsilon$$
Thus,
$$|a_m - a| + |a_m - b| < 2\epsilon$$
or equivalently
$$|a_m - a| + |b - a_m| < 2\epsilon$$
By triangle inequality,
$$2\epsilon > |a_m - a| + |b - a_m| \geq |a_m - a + b - a_m| = |b - a|$$
Hence $a = b$.

\subsection*{Exercise 2.2.7}
(a)
Frequently.

\noindent
(b)
Eventually is stronger.

\noindent
(c)
$a_n$ converges to $L$ if, for every $\epsilon > 0$, all terms $a_n$ eventually fall within the set $(L - \epsilon, L + \epsilon)$.

\noindent
(d)
Not necessarily eventually, but necessarily frequently.

\subsection*{Exercise 2.2.8}
(a)
Yes.

\noindent
(b)
Yes.

\noindent
(c)
No. Consider $\{ 0, 1, 0, 1, 1, 0, 1, 1, 1, 0, 1, 1, 1, 1, 0, \ldots \}$, i.e. we skip larger and larger steps when adding the zeros.

\noindent
(d)
Too lazy.

\section{The Algebraic and Order Limit Theorems}

\subsection*{Exercise 2.3.1}
(a)
Let $\sqrt{x_n} \rightarrow L$. Since $\sqrt{x_n} \geq 0$, then $L \geq 0$ by order limit theorem. Also since $\sqrt{x_n} \leq x_n$ and $x_n \rightarrow 0$, then $L \leq 0$ by order limit theorem. Hence $L = 0$.

\noindent
(b)
Too lazy.

\subsection*{Exercise 2.3.2}
Fuh naw.

\subsection*{Exercise 2.3.3}
Simply use order limit theorem twice. Let $y_n \rightarrow y$.
$$x_n \leq y_n \Rightarrow l \leq y$$
$$y_n \leq z_n \Rightarrow y \leq l$$
Hence $y = l$.

\subsection*{Exercise 2.3.4}
Fuh naw.

\subsection*{Exercise 2.3.5}
Too easy.

\subsection*{Exercise 2.3.6}
Too lazy.

\subsection*{Exercise 2.3.7}
(a)
Let $x_n = n$ and $y_n = -n$.

\noindent
(b)
Impossible.

\noindent
(c)
Let $b_n = \frac{1}{n}$.

\noindent
(d)
Impossible.

\noindent
(e)
Let $a_n = \frac{1}{n}$ and $b_n = n$. The idea is to make the denominator zero, which violates the division theorem.

\subsection*{Exercise 2.3.8}
(a)
Simply apply algebraic limit theorems.

\noindent
(b)
Choose any function with a point-discontinuity.

\subsection*{Exercise 2.3.9}
(a)
We can't use algebraic limit theorem because $a_n$ need not have a limit. Now, to prove the statement, since $a_n$ is bounded, there exists $C \in \mathbb{R}$ such that
$$-C \leq a_n \leq C$$
Thus,
$$-C \cdot b_n \leq a_n \cdot b_n \leq C \cdot b_n$$
$$\lim \left( -C \cdot b_n \right) \leq \lim \left( a_n \cdot b_n \right) \leq \lim \left( C \cdot b_n \right)$$
$$0 \leq \lim \left( a_n \cdot b_n \right) \leq 0$$
Hence, $\lim a_n b_n = 0$.

\noindent
(b)
No, if $b_n$ has nonzero limit, then $a_n b_n$ need not converge.

\noindent
(c)
Too lazy.

\subsection*{Exercise 2.3.10}
(a)
False. If $a_n = b_n = n$, then $\lim (a_n - b_n) = 0$, but $\lim a_n = \lim b_n = \infty$.

\noindent
(b)
True...

\noindent
(c)
True.
$$\lim b_n = \lim \left( a_n + (b_n - a_n) \right)$$
$$= \lim a_n + \lim (b_n - a_n)$$
$$= a + 0 = a$$

\noindent
(d)
True... $a_n$ simply acts as $\epsilon$ here.

\subsection*{Exercise 2.3.11}
Too lazy.

\subsection*{Exercise 2.3.12}
(a)
False. For example, $\max B \in B$ and $a_n = \max B + \frac{1}{n}$.

\noindent
(b)
True. By definition, closed sets contain their limits.

\noindent
(c)
False. Just use the $\sqrt{2}$ example.

\subsection*{Exercise 2.3.13}
(a)
$$\lim_{n \rightarrow \infty} \left( \lim_{m \rightarrow \infty} \frac{m}{m + n} \right) = \lim_{n \rightarrow \infty} 1 = 1$$
$$\lim_{m \rightarrow \infty} \left( \lim_{n \rightarrow \infty} \frac{m}{m + n} \right) = \lim_{m \rightarrow \infty} 0 = 0$$

\noindent
(b)
$\lim_{m, n \rightarrow \infty} \frac{1}{m + n} = 0$ exists. The two iterated limits give the same value 0. $\lim_{m, n \rightarrow \infty} \frac{mn}{m^2 + n^2} = 0$ exists. The two iterated limits also give the same value 0.

\noindent
(c)
I gave up and searched online. Here is the example:
$$a_{mn} = \frac{(-1)^m}{n} + \frac{(-1)^n}{m}$$

\noindent
(d) and (e) Too lazy.

\section{The Monotone Convergence Theorem and Infinite Series}

\subsection*{Exercise 2.4.1}
(a)
It's decreasing and bounded below by $\frac{1}{4}$.

\noindent
(b)
This simply excludes the first term, which does not change its limiting behavior.

\noindent
(c)
$$\lim x_{n + 1} = \lim \frac{1}{4 - x_n}$$
$$\lim x_n = \lim \frac{1}{4 - x_n}$$
$$\lim x_n = \frac{1}{4 - \lim x_n}$$
$$(\lim x_n)^2 - 4 \lim x_n + 1 = 0$$
$$\lim x_n = 2 \pm \sqrt{3}$$
Here, $\lim x_n = 2 - \sqrt{3}$ makes more sense given our starting values.

\subsection*{Exercise 2.4.2}
(a)
We have no information about monotonicity or boundedness, or any condition strong enough.

\noindent
(b)
Too lazy. But either yes or no, after testing for conditions.

\subsection*{Exercise 2.4.3}
(a)
$$x_{n + 1} = \sqrt{2 + x_n}$$
$$\lim x_{n + 1} = \lim \sqrt{2 + x_n}$$
$$\lim x_n = \sqrt{2 + \lim x_n}$$
$$L^2 = 2 + L$$
$$L = 2$$
($L = -1$ does not make sense.)

\noindent
(b)
Too lazy.

\subsection*{Exercise 2.4.4}
(a)
Too lazy.

\noindent
(b)
The closed intervals each have a lower bound. These lower bounds are increasing and bounded above by the first interval's upper bound. Hence, by MCT, the sequence is convergent. Since arbitrary intersection of closed sets is closed, and closed sets contain all their limit points, then the limit of the sequence is inside the intersection.

\subsection*{Exercise 2.4.5+}
Too lazy. Skipped.

\section{Subsequences and the Bolzano-Weierstrass Theorem}

\subsection*{Exercise 2.5.1}
(a)
Impossible. The subsequence is a sequence itself, and since it's bounded, then the subsequence must contain a convergent subsequence. Since this convergent subsequence is a subsequence of the subsequence, then it's also a subsequence of the original sequence.

\noindent
(b)
Simply take a sequence $a_n = 1 - \frac{1}{n}$ for $n$ odd, and $a_n = \frac{1}{n}$ for $n$ even.

\noindent
(c)
Take the sequence $1, 1, \frac{1}{2}, 1, \frac{1}{2}, \frac{1}{3}, 1, \frac{1}{2}, \frac{1}{3}, \frac{1}{4}, 1, \frac{1}{2}, \frac{1}{3}, \frac{1}{4}, \frac{1}{5}, \ldots$.

\noindent
(d)
Same sequence as (c).

\subsection*{Exercise 2.5.2}
(a)
True. If every subsequence converges, then take the subsequence that only excludes the first term. Their tails obviously behave the same way.

\noindent
(b)
True. This is the contrapositive of Bolzano-Weierstrass.

\noindent
(c)
True. If every subsequence converged to the same limit, then the original sequence converges, which is a contradiction.

\noindent
(d)
True. Bounded above by the limit of that particular subsequence.

\subsection*{Exercises 2.5.3 and 2.5.4}
Too lazy. Skipped.

\subsection*{Exercise 2.5.5}
Too lazy.

\subsection*{Exercise 2.5.6}
$$\lim b^{\frac{1}{n}} = b^{\lim \frac{1}{n}} = b^0 = 1$$

\subsection*{Exercise 2.5.7}
Too lazy.

\subsection*{Exercise 2.5.8}
(a)
Kind of obvious. For the infinite case, just let $a_n = \frac{1}{n}$ for $n$ odd, and $a_n = 0$ for $n$ even.

\noindent
(b)
??

\subsection*{Exercise 2.5.9}
Too lazy.

\section{The Cauchy Criterion}

\subsection*{Exercise 2.6.1}
$$|a_N - a| + |a - a_M| < \epsilon_N + \epsilon_M$$
Take the max between $N$ and $M$, call it $T$, to get
$$|a_n - a| + |a - a_m| < 2 \epsilon_T$$
whenever $n, m > T$. This $\epsilon$ can be shrunk indefinitely by increasing $T$, hence
$$|a_n - a + a - a_m| < \epsilon$$
$$|a_n - a_m| < \epsilon$$
for all $\epsilon > 0$.

\subsection*{Exercise 2.6.2}
(a)
Take any convergent alternating series.

\noindent
(b)
Impossible. Cauchy is bounded, hence its subsequences must be bounded.

\noindent
(c)
Impossible. For every term of a divergent monotone sequence, you can find another term strictly greater, later in the sequence. (Or strictly smaller, if it's decreasing.) If that was not the case, then all later terms are equal, which contradicts the fact that it diverges. Anyways, since we can always find greater terms, the Cauchy criterion is violated.

\noindent
(d)
Just take any unbounded sequence with a convergent subsequence, such as alternating $a_n = 0$ and $a_n = n$ for odd/even indices.

\subsection*{Exercise 2.6.3}
(a)
For every $\epsilon$, there exists $n, m, a, b$ such that
$$|x_n - x_m| + |y_a - y_b| < 2\epsilon$$
Take the max of $n, a$ and $m, b$, call them $s, t$. We still have
$$|x_s - x_t| + |y_s - y_t| < 2\epsilon$$
Hence,
$$|x_s - x_t + y_s - y_t| < 2\epsilon$$
$$|(x_s + y_s) - (x_t + y_t)| < 2\epsilon$$

\noindent
(b)
Too lazy.

\subsection*{Exercises 2.6.4+}
Too lazy. Skipped.

\chapter{Basic Topology of $\mathbb{R}$}

\section{}
\section{Open and Closed Sets}

\subsection*{Exercise 3.2.1}
(a)
When taking the $\min$, we assumed the set of $\epsilon$ is finite.

\noindent
(b)
$$O_n = \left( -\frac{1}{n}, \frac{1}{n} \right)$$

\subsection*{Exercise 3.2.2}
(a)
The limit points of $A$ are $-1$ and $1$. The limit points of $B$ are $[0, 1]$.

\noindent
(b)
$A$ and $B$ are neither open nor closed.

\noindent
(c)
Every point of $A$ is isolated. None of the points in $B$ are isolated.

\noindent
(d)
Simply take the union of the limit points found in (a) with the original sets.

\subsection*{Exercise 3.2.3}
(a)
Not open: every $\epsilon$-neighborhood contains irrational numbers. Not closed: every irrational number is a limit point not contained in $\mathbb{Q}$.

\noindent
(b)
Not open: every $\epsilon$-neighborhood contains non-integral numbers. It's closed though.

\noindent
(c)
It's open. Not closed: $0$ is a limit point.

\noindent
(d)
Not open: obvious. Not closed: the series has a limit not contained in the set.

\noindent
(e)
Not open: obvious. Closed: the series is not convergent, hence we need not care whether its limit is contained in the set or not.

\subsection*{Exercise 3.2.4}
(a)
For every $N_\epsilon$ of $\sup A$, there exists $a \in A$ such that $a \in N_\epsilon$. Suppose that this is not the case. Then that means there exists an upper bound smaller than $\sup A$ in the interval $(\sup A - \epsilon, \sup A)$, which contradicts the definition of $\sup A$. Hence, we can contruct a sequence $a_n$ for each smaller $N_\epsilon$, which converges to $\sup A$.

\noindent
(b)
No. There are no elements in the set greater than the supremum, hence the right side of the $\epsilon$-neighborhood will always be floating.

\subsection*{Exercise 3.2.5}
By Theorem 3.2.5, every convergent sequence's limit must be contained in $F$. Every convergent sequence is a Cauchy sequence, hence every Cauchy sequence has a limit in $F$.

\subsection*{Exercise 3.2.6}
(a)
True. We can the arbitrary union of the neighborhoods of each rational number, which gives $\mathbb{R}$.

\noindent
(b)
True. Assuming the closed sets $C_n$ are still bounded, then the sequence $\{\sup C_m \mid m \in \mathbb{N}\}$ is decreasing and bounded below by $\inf C_1$. We argue that $\lim \sup C_m$ must be contained in $\bigcap_{n \in \mathbb{N}} C_n$. Since $\sup C_m \in C_m$ for all $m \in \mathbb{N}$, and $C_a \subseteq C_b$ for $b < a$, then $\sup C_a \in C_b$ for all $b < a$. In other words, each $C_n$ contains all but finitely many elements of $\sup C_m$. Since each $C_n$ is closed, then each $C_n$ contains all of its limit points, which includes the limit of $\sup C_m$. Hence, since $\sup C_m \in C_n$ for all $n \in \mathbb{N}$, then $\sup C_m$ must lie in their intersection as well.

\noindent
(c)
True. Given an element $a$ of a non-empty open set, we can find $N_\epsilon$ around $a$ contained within the set. Since there's always a rational number closer to any real number, then the neighborhood of $a$ must contain a rational number as well. By subset relation, the rational number is also contained within the open set.

\noindent
(d)
False. Simply create a sequence $a_n$ that approaches $\sqrt{2}$ where $a_n \not \in \mathbb{Q}$ for all $n$, and take the union $\{ a_n \mid n \in \mathbb{N} \} \cup \{ \sqrt{2} \}$.

\noindent
(e)
True. It's formed by a sequence of nested closed sets, and so the intersection, which is the Cantor set, must be closed as well.

\subsection*{Exercise 3.2.7}
(a)
Let $a_n$ be a sequence in $L$ with limit $a$. Since each $a_n$ is a limit point of $A$, then we can always find $b_n \in A$ such that $|b_n - a_n| < \epsilon$ for all $\epsilon$. Hence, $b_n$ also approaches $a$. Since $A$ is closed, then it must contain the limit of $b_n$. Hence, $\lim b_n = a$ is a limit point of $A$, hence $a \in L$. Since $a_n$ was an arbitrarily chosen sequence in $L$, then every limit point of $L$ is contained in $L$. Hence $L$ is closed.

\noindent
(b)
If $x \in A$, then since $A$ is closed, we can find a sequence in $A$ approaching $x$, so $x$ is a limit point of $A$. If $x \in L$, then since every limit points of $L$ is a limit point of $A$, then $x$ is a limit point of $A$. To prove the theorem, since every limit point $x$ of $A \cup L$ is a limit point of $A$, then $x \in L \subseteq A \cup L$, hence $A \cup L$ is closed.

\subsection*{Exercise 3.2.8}
(a)
Closed. Closure is always closed.

\noindent
(b)
Open. This is equivalent to $A \cap B^c$, where $B^c$ is obviously open.

\noindent
(c)
Open. $A^c$ is closed, so its union with $B$ is closed. So the whole complement is open.

\noindent
(d)
Closed. This is equivalent to $B \cap (A \cup A^c)$.

\noindent
(e)
Neither. One is open, the other is closed. So their intersection is unclear.

\subsection*{Exercise 3.2.9}
(a)
For the first Morgan equation.
$$x \in \left( \bigcup_{\lambda \in \Lambda} E_\lambda \right)^c$$
$$\Leftrightarrow \neg \left( x \in \bigcup_{\lambda \in \Lambda} E_\lambda \right)$$
$$\underset{\text{def}}{\Leftrightarrow} \neg [(\exists \lambda \in \Lambda)(x \in E_\lambda)]$$
$$\Leftrightarrow (\forall \lambda \in \Lambda) (\neg (x \in E_\lambda))$$
$$\Leftrightarrow (\forall \lambda \in \Lambda) (x \in (E_\lambda)^c)$$
$$\underset{\text{def}}{\Leftrightarrow} x \in \bigcap_{\lambda \in \Lambda} (E_\lambda)^c$$
For the second Morgan equation.
$$x \in \left( \bigcap_{\lambda \in \Lambda} E_\lambda \right)^c$$
$$\Leftrightarrow \neg \left( x \in \bigcap_{\lambda \in \Lambda} E_\lambda \right)$$
$$\underset{\text{def}}{\Leftrightarrow} \neg [(\forall \lambda \in \Lambda)(x \in E_\lambda)]$$
$$\Leftrightarrow (\exists \lambda \in \Lambda) (\neg (x \in E_\lambda))$$
$$\Leftrightarrow (\exists \lambda \in \Lambda) (x \in (E_\lambda)^c)$$
$$\Leftrightarrow x \in \bigcup_{\lambda \in \Lambda} (E_\lambda)^c$$

\noindent
(b)
Given closed sets $A_n$,
$$\bigcap A_n = \left( \bigcup A_n^c \right)^c$$
Since $A_n^c$ is open, then their arbitrary union is open, hence the whole complement is closed, hence $\bigcap A_n$ is closed.

\subsection*{Exercise 3.2.10}
(i)
Impossible. A countable set is a sequence. Since the sequence is bounded by $[0, 1]$, then there exists a convergent subsequence by Bolzano-Weierstrass. The limit of that subsequence is a limit point.

\noindent
(ii)
Possible. Take the rational numbers between $[0, 1]$. None is isolated, and it's countable.

\noindent
(iii)
Impossible. Each isolated point has an $\epsilon$-neighborhood containing no other points. Each neighborhood contains a rational number. Hence, there is a bijection from the set of neighborhoods to the rationals. Hence, the set of neighborhoods is countable, thus the isolated points are countable.

\subsection*{Exercises 3.2.11+}
Too lazy.

\section{Compact Sets}

\subsection*{Exercise 3.3.1}
Since $K$ is non-empty and bounded, it must have a supremum. Since it's closed, it must contain its supremum.

\subsection*{Exercise 3.3.2}
(a)
Not compact. It's not bounded. So take the sequence $a_n = n$.

\noindent
(b)
Not compact. It's not closed. Take any monotone sequence converging to $\sqrt{2}$.

\noindent
(c)
Compact. It's closed and bounded.

\noindent
(d)
Not compact. It's not closed. The sequence itself is convergent, thus its subsequences also converge to the same limit. But the limit is not contained in the sequence.

\noindent
(e)
Not compact. It's not closed. The sequence itself is convergent, thus its subsequences also converge to the same limit. But the limit is not contained in the sequence.

\subsection*{Exercise 3.3.3}
Since $K$ is bounded, every sequence in $K$ has a convergent subsequence by Bolzano-Weierstrass. Since $K$ is closed, the convergent subsequences, which are themselves sequences, must have limits in $K$.

\subsection*{Exercise 3.3.4}
(a)
Definitely closed.

\noindent
(b)
Definitely closed.

\noindent
(c)
Neither.

\noindent
(d)
Neither.

\subsection*{Exercise 3.3.5+}
Too lazy.

\section{Perfect Sets and Connected Sets}

\subsection*{Exercise 3.4.1}
Since $K$ is compact, then its intersection with anything is bounded. Since both $K$ and $P$ are closed, then their intersection is closed. Hence, it's compact.

However, for example, if $K$ is a single point, and $K \subseteq P$, then its intersection with $P$ is a single point, which is isolated, hence it's not perfect.

\subsection*{Exercise 3.4.2}
No. A perfect set has uncountably many elements.

\subsection*{Exercise 3.4.3}
Too lazy.

\subsection*{Exercise 3.4.4}
(a)
It's still compact and perfect.

\noindent
(b)
The smaller versions are $\frac{3}{8}$ of the bigger versions, and their volume is half. Hence, when you scale by $\frac{8}{3}$, your volume gets scaled by $2$. The dimension is $\log_{\frac{8}{3}} 2 \approx 0.706695052611$. The length is still zero.

\subsection*{Exercise 3.4.5}
Since $U, V$ are open and disjoint, then around each point $u \in U$, there's a neighborhood $N_\epsilon(u)$ which contains no point of $V$. Hence, no sequence of $V$ converges to $u \in U$. And vice-versa. Hence, their are disconnected. Since $A$ and $B$ are subsets of $U, V$, then simply elevate their sequences into the space of $U, V$, hence proving that $A, B$ are disconnected.

\subsection*{Exercise 3.4.6}
Suppose $E$ is connected. Then for every disjoint sets $A, B$ satisfying $A \cup B = E$, by definition of connectedness, we know
$$\bar{A} \cap B \neq \varnothing$$
or
$$\bar{B} \cap A \neq \varnothing$$
In the first case, since $\bar{A} = A \cup L_A$, but $A \cap B = \varnothing$, then it must be that $L_A \cap B \neq \varnothing$. Hence, there exists a limit point of $A$ contained in $B$. The second case follows a similar procedure.

The reverse direction is obvious. If there's a convergent sequence in $A$ with limit in $B$, then $\bar{A} \cap B \neq \varnothing$. The case for a convergence sequence in $B$ follows a similar procedure.

\subsection*{Exercise 3.4.7}
(a)
Between every two distinct real number $x, y$ lies an irrational number $x < c < y$. Clearly, $E \cap (-\infty, c)$ and $E \cap (c, \infty)$ are disconnected components of $E$ which contain $x$ and $y$ respectively.

\noindent
(b)
Yes. The argument is identical to the rational numbers'.

\subsection*{Exercises 3.4.8+}
Too lazy.

\section{Baire's Theorem}

\subsection*{Exercise 3.5.1}
$$A = \bigcap A_n$$
where $A_n$ are open. Then,
$$A^c = \left( \bigcap A_n \right)^c = \bigcup A_n^c$$
which is a countable union of closed sets.

\subsection*{Exercise 3.5.2}
(a)
Countable.

\noindent
(b)
Finite.

\noindent
(c)
Finite. Explanation from online: for each rational number $q$, create the singleton set $\{ q \}$. Each singleton is clearly $G_\delta$, since it's the countable intersection of neighborhoods around $q$. We can do the countable union of the singletons to get $\mathbb{Q}$. However, $\mathbb{Q}$ is not $G_\delta$ (as we'll see later). Hence, the countable union of $G_\delta$ is not necessarily $G_\delta$.

\noindent
(d)
Countable.

\subsection*{Exercise 3.5.3}
(a)
Take the intersection of open sets of the form $(a - \epsilon, b + \epsilon)$ for $\epsilon > 0$.

\noindent
(b)
Take the intersection of open sets of the form $(a, b + \epsilon)$ for $\epsilon > 0$. Or take the union of closed sets of the form $[a + \epsilon, b]$ for $\epsilon > 0$.

\noindent
(c)
Take the union of rational singletons to get $\mathbb{Q}$. Its complement, the irrationals, is thus $G_\delta$ by Exercise 3.5.1. Another way to prove the irrationals is to consider open sets of the form $(-\infty, q) \cup (q, \infty)$, where $q$ is rational. Since $\mathbb{Q}$ is countable, we can simply take the countable intersection of all these open sets, which effectively puncture the number line at the precise location of rationals.

\subsection*{Exercise 3.5.4}
Take a point $g_1 \in G_1$. Since it's open, I can find $N_\epsilon (g_1) \subseteq G_1$. Since $G_2$ is dense, then there exists a point $g_2 \in G_2$ such that $g_2 \in (g_1 - \epsilon, g_1)$. Since $G_2$ is open, then I can find $N_\delta (g_2) \subseteq G_2$. Clearly, I can find one that satisfies $N_\delta (g_2) \subseteq N_\epsilon (g_1)$. Now, simply let $I_1$ be a closed set inside $N_\epsilon (g_1)$, and $I_2$ be a closed set inside $N_\delta (g_2)$. Clearly, I can still make them keep their subset relation. We continue this process to create every $I_n$. This intersection must be non-empty by Nested Interval Property. And since $I_n \subseteq G_n$ for all $n$, then the intersection of $G_n$ must also be non-empty.

\subsection*{Exercises 3.5.5+}
??

\subsection*{Exercise 3.5.8}
Suppose $\bar{E}^c$ is dense. Then for every neighborhood $N_\epsilon (x) \subseteq \bar{E}$, you can $y \in \bar{E}^c$ such that $y \in N_\epsilon$. However, $y \not \in \bar{E}$. Hence, $N_\epsilon$ cannot be an open subset of $\bar{E}$. Since $x$ was arbitrarily chosen, it follows that $\bar{E}$ does not contain non-empty open subsets, hence $E$ is nowhere-dense.

Suppose $E$ is nowhere-dense. Since $\bar{E}$ contains no open sets, then for every neighborhood $N_\epsilon (x)$ for $x \in \mathbb{R}$, we want $N_\epsilon (x) \not \subseteq \bar{E}$. Hence, for each such neighborhood, there must exist a point $y \in N_\epsilon (x)$ such that $y \not \in \bar{E}$, i.e. $y \in \bar{E}^c$. Thus, we are able to approach every point of $\mathbb{R}$ using a sequence found in $\bar{E}^c$. Hence, $\bar{E}^c$ is dense.

\subsection*{Exercise 3.5.9}
(a)
Nowhere dense.

\noindent
(b)
Nowhere dense.

\noindent
(c)
Dense.

\noindent
(d)
Nowhere dense.

\subsection*{Exercise 3.5.10}
$$\bigcup E_n = \mathbb{R}$$
$$\Rightarrow \bigcup \bar{E}_n = \mathbb{R}$$
$$\Rightarrow \left( \bigcup \bar{E}_n \right)^c = \varnothing$$
$$\Rightarrow \bigcap \bar{E}_n^c = \varnothing$$
Then, ???

\chapter{Functional Limits and Continuity}

\section{}
\section{Functional Limits}

\subsection*{Exercise 4.2.1}
Too lazy.

\subsection*{Exercise 4.2.2}
Too lazy.

\subsection*{Exercise 4.2.3}
We already know it's discontinuous at every rational number.

\subsection*{Exercise 4.2.4}
(a)
We want $\frac{1}{[[x]]} \leq \frac{6}{10}$. Thus $x = 2$. Anything lower will become $\frac{1}{1}$ which is greater than $\frac{6}{10}$. For $\frac{1}{[[x]]} \geq -\frac{4}{10}$, everything works, so $x = \infty$. Hence, we must choose $\delta = 10 - 2 = 8$, ignoring the infinity.

\noindent
(b)
Too lazy.

\noindent
(c)
$\epsilon = \frac{1}{9} - \frac{1}{10}$

\subsection*{Exercise 4.2.5}
Too lazy.

\subsection*{Exercise 4.2.6}
(a)
True. Everything within distance $\delta$.

\noindent
(b)
False. We can create a point discontinuity.

\noindent
(c)
True. This is a property of the composition of continuous functions.

\noindent
(d)
False. Consider $f = \frac{1}{x}$ and $g = x^2$ on the domain $(0, \infty)$.

\subsection*{Exercise 4.2.7}
Use the Squeeze Theorem. For any sequence $x_n \rightarrow c$, we have $f(x_n)$ and $g(x_n)$ which are sequences as well. We know $g(x_n) \rightarrow 0$. We know $-M \leq f(x_n) \leq M$. So we have
$$-Mg(x_n) \leq f(x_n)g(x_n) \leq Mg(x_n)$$
$$\lim -Mg(x_n) \leq \lim f(x_n)g(x_n) \leq \lim Mg(x_n)$$
$$-M\lim g(x_n) \leq \lim f(x_n)g(x_n) \leq M\lim g(x_n)$$
$$0 \leq \lim f(x_n)g(x_n) \leq 0$$
Hence $f(x_n)g(x_n) \rightarrow 0$ as $x_n \rightarrow c$. Since the sequence $x_n$ was chosen arbitarily, by definition we have $\lim_{x \rightarrow c} f(x)g(x) = 0$.

\subsection*{Exercise 4.2.8}
Too lazy.

\subsection*{Exercise 4.2.9}
(a)
Given $M > 0$, choose $\delta = \frac{1}{2\sqrt{M}}$. Then the smallest is achieved when either $f(x) = f(\delta) = 4M$ or $f(x) = f(-\delta) = 4M$. Clearly, $4M > M$.

\noindent
(b)
$\lim_{x \rightarrow \infty} f(x) = L$ means that for all $\epsilon > 0$, we can find $M > 0$ such that whenever $x > M$, then $|f(x) - L| < \epsilon$. Too lazy to show.

\noindent
(c)
$\lim_{x \rightarrow \infty} f(x) = \infty$ means that for all $M > 0$, we can find $N > 0$ such that whenever $x > N$, then $f(x) > M$. Too lazy for example.

\subsection*{Exercise 4.2.10}
(a)
$\lim_{x \rightarrow a+} f(x) = L$ means that for all $\epsilon > 0$, we can find $\delta > 0$ such that whenever $0 < x - a < \delta$, then $|f(x) - L| < \epsilon$. Too lazy for left side.

\noindent
(b)
We can choose the same $\epsilon$, then choose the smallest $\delta$. We combine $0 < x - a < \delta$ and $-\delta < x - a < 0$ to give $0 < |x - a| < \delta$ that satisfies $|f(x) - L| < \epsilon$.

\subsection*{Exercise 4.2.11}
This is simply an application of the "Sequence" Squeeze Theorem to the sequence definition of functional limits.

\section{Continuous Functions}

\subsection*{Exercise 4.3.1}
Too lazy.

\subsection*{Exercise 4.3.2}
(a)
The constant function.

\noindent
(b)
The linear function $y = x$. It is not onetinuous, since things at a distance of $1$ are not equal to the center.

\noindent
(c)
The linear function $y = 2x$. It is not equaltinuous, since distinct $y$'s separated by $\epsilon$ required $\delta \leq \frac{\epsilon}{2}$.

\noindent
(d)
They are equivalent. If we can find $\delta > 0$ that satisfies the inequality in continuity, then a smaller $\delta$ will still satisfy the inequality, so we can choose arbitrary $\delta < \epsilon$ that still works, thus satisfying lesstinuity. Likewise, if we find $\delta < \epsilon$ for lesstinuity, then we effectively found $\delta$ for continuity, regardless of whether it's smaller or greater than $\epsilon$.

\subsection*{Exercise 4.3.3}
(a)
Suppose we have $h(x) = g(f(x))$.

Given $\epsilon > 0$, using the continuity of $g$, we can find $\delta_g$ such that $|f(x) - f(c)| < \delta_g$ implies $|g(f(x)) - g(f(c))| < \epsilon$, where $f(x)$ and $f(c)$ are points in the domain of $g$. But they are also points in the range of $f$. Hence, we can treat $\delta_g$ as $\epsilon_f$. Since we have $|f(x) - f(c)| < \delta_g = \epsilon_f$, using the continuity of $f$, we can find $\delta > 0$ such that $|x - c| < \delta$ implies $|f(x) - f(c) < \epsilon_f$. Hence, we have a logical chain:
$$|x - c| < \delta \quad \Rightarrow \quad |f(x) - f(c)| < \epsilon_f = \delta_g \quad \Rightarrow \quad |g(f(x)) - g(f(c))| < \epsilon$$
Hence, given any $\epsilon$, we can find such a $\delta$ that creates this implication. Notice that $g(f(x))$ is precisely equal to $h(x)$, so replacing it in the implication gives us the desired result. QED.

\noindent
(b)
$$x_n \rightarrow c \quad \Rightarrow \quad f(x_n) \rightarrow f(c) \quad \Rightarrow \quad g(f(x_n)) \rightarrow g(f(c))$$
The main difference from (a) is that we do not need to backtrack to show existence of anything.

\end{document}
