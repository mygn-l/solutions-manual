\documentclass{book}

\usepackage{amsmath}
\usepackage{amsfonts}
\usepackage{amssymb}
\usepackage{amscd}
\usepackage{amsthm}
\usepackage{graphics}
\usepackage{graphicx}
\usepackage{tikz}
\usepackage{tikz-cd}
\usepackage{bm}

\newtheorem{theorem}{Theorem}
\newtheorem{definition}{Definition}
\newcommand{\bracenom}{\genfrac{\lbrace}{\rbrace}{0pt}{}}

\graphicspath{ {./} }

\title{Abstract Algebra Herstein - Solutions Manual}
\author{Mingruifu Lin}
\date{September 2023}

\begin{document}

\maketitle

\tableofcontents

\chapter{Preliminary Notions}

\section{Set Theory}

\subsection*{Problem 1}
(a)
We expand the definitions $A \subseteq B$ and $B \subseteq C$:
$$x \in A \Rightarrow x \in B$$
$$x \in B \Rightarrow x \in C$$
Suppose $x \in A$. Then by modus ponens, $x \in B$. Again by modus ponens, $x \in C$. Hence, by conditional proof, $x \in A \Rightarrow x \in C$. This is the definition of $A \subseteq C$.

\noindent
(b)
Suppose $x \in A \cup B$. We check two cases. If $x \in A$, then $x \in A$. If $x \in B$, then using $B \subseteq A$ and modus ponens, we get $x \in A$, hence $x \in A$. Thus $A \cup B \subseteq A$.

For the reverse direction, suppose $x \in A$. Then $x \in A \cup B$ by disjunction introduction. Thus $A \subseteq A \cup B$. Hence proven.

\noindent
(c)
Too lazy. Disjunctions are always tedious, as seen previously.

\subsection*{Problem 2}
(a)
For intersection:
$$x \in A \cap B$$
$$\Leftrightarrow x \in A \text{ and } x \in B$$
$$\Leftrightarrow x \in B \cap A$$
For union: Too lazy. Again, too many disjunctions.

\noindent
(b)
Same idea as (a). Simply apply conjunction elimination twice, then conjunction introduction twice.

\subsection*{Problem 3}
Here we gooooo.

Suppose $x \in A \cup (B \cap C)$. If $x \in A$, then $x \in A \cup B$ and $x \in A \cup C$, hence $x \in (A \cup B) \cap (A \cup C)$. Otherwise, $x \in B \cap C$, so $x \in B$ and $x \in C$, thus $x \in B \cup A$ and $x \in C \cup A$, hence $x \in (A \cup B) \cap (A \cup C)$. Hence $A \cup (B \cap C) \subseteq (A \cup B) \cap (A \cup C)$.

For the reverse direction, suppose $x \in (A \cup B) \cap (A \cup C)$. Then $x \in A \cup B$ and $x \in A \cup C$. If $x \in A$, then $x \in A \cup (B \cap C)$. If $x \in B$, then, in order to satisfy the second statement, either $x \in A$, which we've seen, or $x \in C$. In the latter case, we thus have $x \in B \cap C$, hence $x \in (B \cap C) \cup A$. In all cases, we have $x \in A \cup (B \cap C)$. Hence $(A \cup B) \cap (A \cup C) \subseteq A \cup (B \cap C)$. Hence proven.

\subsection*{Problem 4}
(a)
$$x \in (A \cap B)'$$
$$\Rightarrow x \not \in A \cap B$$
Using the fact $\neg(A \land B) = \neg A \lor \neg B$:
$$\Rightarrow x \not \in A \text{ or } x \not \in B$$
$$\Rightarrow x \in A' \text{ or } x \in B'$$
$$\Rightarrow x \in A' \cup B'$$
Too lazy to prove the reverse direction.

\noindent
(b)
Too lazy.

\subsection*{Problem 5}
Idk what I'm allowed to do bro.

\subsection*{Problem 6}
Include or exclude an element.

\subsection*{Problem 7}
At least $39\%$ like both. At most $63\%$ like both.

\subsection*{Problems 8, 9}
Too lazy, skipped.

\subsection*{Problem 10}
(a)
No. The common ancestor could be different for each pair.

\noindent
(b)
No. For example, one at far left, one in middle, one at far right.

\noindent
(c)
Yes.

\noindent
(d)
Yes.

\noindent
(e)
No. Equivalence relation must be reflexive.

\noindent
(f)
Yes.

\subsection*{Problem 11}
(a)
The reflexive property guarantees the existence of equivalence classes on non-empty sets, whereas the other properties do not.

\noindent
(b)
Idk. Maybe $a \in R \Rightarrow a \sim a$.

\subsection*{Problems 12 and 13}
Too lazy.

\section{Mappings}

\subsection*{Problem 1}
(a)
Onto, but not one-to-one.

\noindent
(b)
Both onto and one-to-one. The inverse image is $t \sigma^{-1} = \sqrt{t}$.

\noindent
(c)
Neither onto nor one-to-one.

\noindent
(d)
One-to-one, but not onto.

\subsection*{Problem 2}
Simply take $f(s \times t) = t \times s$.

\subsection*{Problem 3}
Too lazy. Seems obvious.

\subsection*{Problem 4}
(a)
Any bijective function has an inverse, which is also a bijection.

\noindent
(b)
Simply take the composition of the bijection.

\subsection*{Problem 5}
???

\subsection*{Problem 6}
This is akin to Cantor's diagonal argument. In the original argument, we create a real number which differs from every listed real number by a single digit, hence it is not in the list. Here, the idea is similar.

Suppose I have a bijection $f: S \rightarrow S^*$, where each $s \in S$ is mapped to a subset $f(s) \in S^*$. Let me construct the set $B = \{ s \in S \mid s \not \in f(s) \}$. In other words, this is the set of elements which are not contained in the subset associated with them. As you see, $B$ differs from $f(s)$ by the single element $s$ for each $f(s)$. If $f(s)$ contains $s$, then $B$ does not contain $s$, and if $f(s)$ does not contain $s$, then $B$ contains $s$. Since for all bijections $f$, we can such a set $B$, hence there exists no bijection between $S$ and $S^*$.

\subsection*{Problem 7}
There are $n!$ ways to permute $n$ objects.

\subsection*{Problem 8}
(a) and (b)
??

\noindent
(c)
For (a), as we learned in real analysis, you can map $[0, 1)$ to $\mathbb{R}$, so you repeat the procedure for every $[n, n + 1)$ for $n \in \mathbb{Z}$. This is onto, but not one-to-one. For (b), simply map $\mathbb{R}$ to $(0, 1)$, which is one-to-one, but not onto.

\subsection*{Problem 9}
(a)
Using the real numbers again, let $\sigma: \mathbb{R} \rightarrow \mathbb{R}$ and $\tau: (0, 1) \rightarrow \mathbb{R}$, but the range of $\sigma$ is $(0, 1)$.

\noindent
(b)
Just some domain BS. The first function must be one-to-one, but its range may not cover the entire domain of the second function. So you can do whatever you want to the things outside of the domain.

\subsection*{Problem 10}
Classic real analysis exercise. Skipped.

\subsection*{Problem 11}
(a)
Obvious?

\noindent
(b)
Too easy.

\noindent
(c)
Again, domain BS.

\subsection*{Problem 12}


\end{document}
