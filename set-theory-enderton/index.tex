\documentclass{book}

\usepackage{amsmath}
\usepackage{amsfonts}
\usepackage{amssymb}
\usepackage{amscd}
\usepackage{amsthm}
\usepackage{graphics}
\usepackage{graphicx}
\usepackage{tikz}
\usepackage{tikz-cd}
\usepackage{bm}

\newtheorem{theorem}{Theorem}
\newtheorem{definition}{Definition}
\newcommand{\bracenom}{\genfrac{\lbrace}{\rbrace}{0pt}{}}

\graphicspath{ {./} }

\title{Set Theory Enderton - Solutions Manual}
\author{Mingruifu Lin}
\date{September 2023}

\begin{document}

\maketitle

\tableofcontents

\chapter{Introduction}

\subsection*{Exercise 1}
(a)
Both $\in$ and $\subseteq$.

\noindent
(b)
Only $\subseteq$.

\noindent
(c)
Only $\subseteq$.

\noindent
(d)
Only $\in$.

\noindent
(e)
None works.

\subsection*{Exercise 2}
The empty set contains no element, whereas the others both contain something. The second set contains the empty set, but the third does not. Hence, they are pairwise unequal.

\subsection*{Exercise 3}
$$a \in \mathcal{P}(B) \Rightarrow a \subseteq B \Rightarrow a \subseteq C \Rightarrow a \in \mathcal{P}(C)$$
where the second implication comes from the transitivity of the subset relation.

\subsection*{Exercise 4}
$$x, y \in B \Rightarrow \{x, y\}, \{x\} \in \mathcal{P} (B) \Rightarrow \{ \{x, y\}, \{x\} \} \in \mathcal{P}\mathcal{P}(B)$$

\subsection*{Exercise 5}
The rank of $\{\{\varnothing\}\}$ is 2. The rank of $\{\varnothing, \{\varnothing\}, \{\varnothing, \{\varnothing\}\}\}$ is 3.

\subsection*{Exercise 6}
I don't understand the question.

\subsection*{Exercise 7}
Fuh naw.

\chapter{Axioms and Operations}

\subsection*{Exercise 1}
Set of integers divisible by 180.

\subsection*{Exercise 2}
Let $A = \{\{1\}, \varnothing\}$ and $B = \{\{1\}\}$.

\subsection*{Exercise 3}

\end{document}
